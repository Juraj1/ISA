\documentclass[a4paper,12pt]{article}
\usepackage[czech]{babel}
\usepackage[utf8]{inputenc}
%\usepackage[unicode]{hyperref}
\usepackage{url}
\usepackage[left=2cm,text={17cm,24cm},top=3cm]{geometry}
%\hypersetup{pdfborder={0 0 0 [0 1]}}
\usepackage{mdwlist}
\usepackage{times}
\usepackage{epic}
\usepackage{graphics}
\usepackage{graphicx}
\usepackage{multirow}
\usepackage{pdflscape}
\usepackage{cite}

\usepackage{xcolor}
\usepackage[colorlinks = true,
linkcolor = black,
urlcolor  = blue,
citecolor = black,
anchorcolor = blue,
unicode]{hyperref}
\newcommand{\MYhref}[3][blue]{\href{#2}{\color{#1}{#3}}}%


\begin{document}
	\pagenumbering{Alph}
		\begin{titlepage}
			\begin{center}
				\Huge
				\textsc{Vysoké učení technické v~Brně\\
					\huge Fakulta informačních technologií\\}
				
				\vspace{\stretch{0.382}}
				
				\Large Síťové aplikace a správa sítí\\
				\Huge 	Jednoduchý monitorovací nástroj protokolů CDP a LLDP
				
				\vspace{\stretch{0.618}}
			\end{center}
			\Large {\today \hfill Jiří Zahradník}
		\end{titlepage}
	\pagenumbering{arabic}
	
	\pagebreak
	
	\tableofcontents
	
	
	\pagebreak
	
	\section{Úvod}
		Tento monitorovací nástroj sleduje packety dvou protokolů. Konkrétně proprietárního
		Cisco Discovery Protocol(CDP)  a průmyslového standardu umožňujícím komunikaci mezi zařízeními napříč výrobci, Link Layer Discovery Protocol(LLDP).
		
	\section{Zapouzdření do ethernetového rámce}
		Tyto discovery protokoly využívají dvou ethernetových rámců. LLDP využívá rámec Ethernet II, narozdíl od CDP, který využívá Ethernet 802.3 s LLC/SNAP \cite{wireshark:ethernet}.
		Tyto dva rámce jsou si až na pár bytů podobné. A to, že Ethernet II, narozdíl od Ethernetu 802.3, má na 13. a 14. bytu typ dat obsažených v payloadu. Ethernet 802.3 má na 13. a 14. bytu délku dat obsažených v payloadu.
		
		\begin{figure}[h]	
			\begin{center}
				\scalebox{0.35}{\includegraphics{etherVsEther.jpg}}
				\caption{Rozdíl mezi rámci Ethernet II a Ethernet 802.3}
				\label{pic:etherVsEther}
			\end{center}
		\end{figure}
		
		Rozlišení mezi těmito rámci není komplikované. Ethernetový rámec nesmí mít více než 1500 bytů dat. Z toho vyplývá, že pokud je v poli type/length hodnota menší nebo rovno 1500, pak je to rámec 802.3. V opačném případě je to rámec Ethernet II \cite{wireshark:ethernet}. 
	
	\section{Protokoly}	
		\subsection{Cisco Discovery Protocol}\label{cdp}
			Cisco Discovery Protocol je proprietární protokol vyskytující se na linkové vrstvě \cite{wiki:cdp} ISO/OSI modelu \cite{wiki:iso/osi}. Cisco zařízené posílají CDP oznámení na muticastovou adresu \texttt{01:00:0c:cc:cc:cc} z každého připojeného rozhraní. V základu, stejně jako v této aplikaci, jsou oznámení odesílaná každých 60 sekund skrz rozhraní podporující Subnetwork Access Protocol(SNAP). Každé Cisco zařízení si udržje tabulku, do které si ukládá informace o svých sousedech obsaženy právě v CDP oznámeních. U každého záznamu je hodnota \texttt{Time To Live}, která určuje, po jakou dobu má být záznam udržován v tabulce.
			
			\subsubsection{Struktura CDP}
				CDP je, krom pevně daných bytů hlavičky, tvořen datovými strukturami \texttt{TLV (Time Length Value)}~\cite{technion:cdp} jak je možno vidět na v tabulce~\ref{cdp:frameFormat}.
				
		
				\begin{table}[h]
					\begin{center}
						\caption{Struktura protokolu CDP}\label{cdp:frameFormat}
						
						\scalebox{0.75}{
							\begin{tabular}{| r | r | r | r |} \hline
								\texttt{Version (1 byte)} & \texttt{Time To Live (1 byte) } & 	\texttt{Checksum (2 bytes)} & \texttt{Time Length Value fields} \\ \hline 
							
							\end{tabular}
						}
						
						
					\end{center}
				\end{table}
				
				Jak můžeme vidět, tak hlavička má 32 bitů, přičemž prvních 8 označuje verzi protokolu, následuje 8 bitů pro \texttt{Time To Live} a poté pokračuje standardní IP kontrolní součet pro kontrolu validity packetu. \texttt{Type} označuje typ hodnoty, která je zapsána v hodnotě \texttt{Value}. \texttt{Length} určuje délku celého TLV pole, takže délka hodnoty \texttt{Value} je o dva byty kratší. Na toto je třeba dávat pozor při implementaci.
				
				\begin{table}[h]
					\begin{center}
						\caption{Struktura pole TLV}\label{cdp:tlvFormat}
						
						\scalebox{0.75}{
							\begin{tabular}{| l | r | r |} \hline
								\texttt{Type (2 bytes)} & \texttt{Length (2 bytes) } & 	\texttt{Value(Length - 4 bytes)} \\ \hline 
								
							\end{tabular}
						}
						
						
					\end{center}
				\end{table}
				
				Hodnoty, které můžou TLV pole nabývat, \href{http://www.cs.technion.ac.il/Courses/Computer-Networks-Lab/projects/spring2003/cdp2/web\_cdp2/web\_cdp2/cdp2\_report.htm}{je možno vidět zde}.
				
				
				
		\subsection{Link-Layer Discovery Protocol}\label{lldp}
			Link-Layer Discovery Protocol je protokol druhé síťové vrstvy využívaný ke stejným účelům jako CDP. Narozdíl od CDP je ale nezávislý na výrobci a detailně je popsán ve standardu 802.1AB \cite{802.1AB}. Identifikace protokolu v ethernetovém rámci je jednoduchá, jednat hodnota typu je 0x88CC a také cílová MAC adresa ethernetového rámce patří do skupiny multicastových adres, které síťové prvky splňující standard 802.1D neposílají dále. Pomocí tohoto protokolu můžeme zjistit následující informace \cite{wiki:lldp}:
			
			\begin{itemize}
				\item Jméno a popis systému
				\item Jméno a popis portu
				\item Jméno VLAN
				\item Schopnosti systému
				\item Fyzickou adresu
				\item Informace o napájení
				\item Agregace linek
			\end{itemize}
			
			\subsubsection{Struktura LLDP}
				LLDP vypadá podobně jako CDP, avšak v některých věcech se liší. Stejně jako CDP se skládá z datových struktur typu TLV. Povinná TLV jsou: \texttt{Chassis ID}, \texttt{Port ID}, \texttt{Time To Live} a \texttt{End of LLDPDU}, přičemž \texttt{End of LLDPDU} jsou dva oktety plné nul. Formát TLV lze vidět na obrázku \ref{lldp:tlvFormat}. Zde je jeden z rozdílů ve srovnání s CDP. Pole \texttt{Length} obsahuje hodnotu reprezentující počet oktetů pole \texttt{Value}. Implementace se sice zde mírně zjednodušší, ale kvůli faktu, že pole \texttt{Type} a \texttt{Length} nejsou zarovnaná, jak můžeme vidět na obrázku \ref{pic:lldpTlvStructure}, je třeba použít bitových posunů nebo bitových polí.
				
				\begin{figure}[h]	
					\begin{center}
						\scalebox{0.6}{\includegraphics{lldpStructure.png}}
						\caption{Struktura LLDP packetu}
						\label{pic:lldpStructure}
					\end{center}
				\end{figure}
				
				\begin{figure}[h]	
					\begin{center}
						\scalebox{0.6}{\includegraphics{lldpTlvStructure.png}}
						\caption{Struktura LLDPDU}
						\label{pic:lldpTlvStructure}
					\end{center}
				\end{figure}
			
	\section{Aplikace}
	\subsection{Návrh}
		Aplikace je navrhnuta jako vícevláknová, kde hlavní vlákno pomocí knihovny \texttt{pcap} zachytává
			
	\pagebreak	
	\listoffigures
	\listoftables
	\pagebreak

	\bibliographystyle{dokumentace}
	\bibliography{dokumentace}
\end{document}